%!TeX root=../Dissertation.tex
%!TeX bibfile=./synthesis.bib

\chapter{System creation and setup}
\section{The host machine}
It was decided as a better test of the ability of Docker to reduce strain on a system, than an older machine with limited resources would be used, over a more modern and very capable machine. This was to better represent the target group of this research; that being SMEs that may have aging hardware, attempting to get the most out of what they currently have available.

\section{Servers}
\subsection{Operating system}
Both systems were created from scratch, using the latest LTS versions of Ubuntu Server (20.04, Focal Fossa Server \citep{UbuntuServerDocumentation}).

For VMware, the images were downloaded directly from the Ubuntu Website and the virtual machines were created using VMware's workstation Virtual Machine creation wizard, whereas for Docker, the Ubuntu Server image was taken from Ubuntu's official Docker Image, hosted on Docker Hub \citep{UbuntuDockerHub}. Each docker image was then created using a Dockerfile, which lays out which version to use; in this case, ubuntu:latest.

\subsection{Software}
The servers were created to use the software and topology laid out in the requirements in the analysis, (section \ref{Requirements:infrastructure}).

For both VMware and Docker this meant using the Advanced Package Tool (which is the default package manager on ubuntu) to install the various software needed.

However, for VMware, this was done once each server was installed and booted. On Docker, through the use of Dockerfiles, this can integrated directly into the Docker image, along with configuration files. An example of a Dockerfile for DNS server 1 is shown in figure \ref{fig:dockerfileexample} below. This shows how the base image is selected, along with the addition of the software and tools necessary to make bind9 work. The `COPY' command takes the completed configuration files and places them in the correct folders.
\begin{figure}[h]
\caption{}
\label{fig:dockerfileexample}
\begin{minted}{dockerfile}
FROM ubuntu:latest

RUN apt-get update \
  && apt-get install -y \
  bind9 \
  bind9utils \
  bind9-doc


COPY named.conf.options /etc/bind/
COPY named.conf.local /etc/bind/
COPY db.intranet.co.uk /etc/bind/
COPY db.72.168.192.in-addr.arpa /etc/bind/

CMD ["/bin/bash", "-c", "while :; do sleep 10; done"]

\end{minted}

\end{figure}

\section{Client}
\subsection{Operating System}
The client machine was created using VMware workstation pro, using Ubuntu's latest LTS desktop version (20.04, Focal Fossa Desktop \citep{UbuntuDesktopDocumentation})

The same client machine was used for both the VMware system and the Docker System. This was to ensure that the client used the same amount of resources on the host machine for both systems, as in a real environement, the clients would be remote devices on the network. By using the same virtual machine as the client in both tests, any substantial impact to the outputs of the tests should be limited.

The Client machine was configured to use up to 4GB of RAM, and up to two cores using VMware Workstation Pro.

\section{The Network}

\subsection{Software}

\chapter{Methods}

\chapter{Testing \& Benchmarks}

%the idle benchmarks didn't have the client machine running!