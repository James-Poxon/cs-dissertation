%!TeX root=../Dissertation.tex

\chapter{Abstract}
Virtualisation is a standard and de-facto technology within server operations, but virtualisation has it's pitfalls, with one of those being the large overheads introduced due to having to virtualise a whole operating system. One possible solution to this problem is containerisation. Containers run without the need for a virtualised operating system, instead integrating directly with the host machine's operating system. This effectively removes a large portion of the processing required to run each instance of a server, which not only makes servers easier to run, but theoretically could improve the speed of said servers.

An in-depth analysis into Virtual Machines and Containers was completed to ensure understanding of the problem domain before delving into the practical part of research. When investigating comparison work of container and virtual machines, it became apparent that not enough research was done to accurately test the real performance benefits of container based server operation. Most comparison work was specifically at raw hardware performance and utilisation, which whilst useful, didn't give results relevant to real-world performance. As a result, we built two separate but logically identical computer networks using A primary DNS, two secondary DNS, DHCP, Web, and MySQL servers. One network used VMware Workstation Pro (Virtualisation) to host the servers, and the other used Docker (Containerisation) to host the servers. The network topology was designed to be reflective of a possible real-world internal network that we may expect to see in an SME (Small and Medium-sized Enterprise). Four separate benchmarks were then designed, using a mix of different techniques and software, such as JMeter and Sysbench, to accurately test every server on the network.

The performance improvements found when moving from VMware to Docker were very impressive. In some cases, Docker produced near double the output that VMware did. We also found that Docker was much more stable, with VMware's performance fluctuating considerably. This was most likely due to memory swapping. This is because one of our tests showed that RAM usage was extremely high on the host machine of the VMware system. The Docker system used far less RAM, and so didn't rely on memory swapping.

Organisations looking to get extra performance out of ageing hardware could use containers as an alternative to virtual machines for their server infrastructure. However, that Docker is not intended to be used in the way we are applying it, and as a result, it was decided that to better support containers in the future and to support the transition from Virtualisation to Containerisation, a better, container-based solution for server management should be developed. The benefits of container-based servers are clear to see as a result of this research, but we need to have a more user-friendly, and server-focused platform that can match the equivalent server-focused virtual machine based platforms that already exist.