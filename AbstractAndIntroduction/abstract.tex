%!TeX root=../Dissertation.tex

\chapter{Abstract}
Virtualisation is a standard and de-facto technology within server operations, but virtualisation has it's pitfalls, with one of those being the large overheads introduced from virtualisation of a whole operating system. A possible solution to this problem is containerisation. Containerisation runs without a virtualised operating system, instead integrating directly with the host machine's operating system. This effectively removes the OS-component processing overhead required to run each instance of a server, which not only makes servers use less system resources, but theoretically could improve the speed of said servers.

An in-depth analysis into Virtual Machines and Containers was completed to ensure understanding of the problem domain before delving into comparison work of container and virtual machines. It became apparent that most previously published comparison work was specifically aimed at raw hardware performance and utilisation, which whilst useful, doesn't give results relevant to real-world performance.

Two separate but topologically identical computer networks were built. Said topology implemented a primary DNS, two secondary DNS, DHCP, Web, and MySQL servers. One network used VMware (Virtualisation), and the other used Docker (Containerisation) to host the servers. The network topology was designed to be reflective of a possible real-world internal network that we may expect to see in an SME (Small and Medium-sized Enterprise). Four separate benchmarks were then designed, using a mix of different techniques and software, such as JMeter and Sysbench, to accurately test every part of the network.

 Impressive and substantial performance improvements found when moving from VMware to Docker. In some cases, Docker produced near double the output that VMware did. Docker was also found to be far more stable than VMware.

Organisations looking to get extra performance out of ageing hardware could potentially use containers as an alternative to virtual machines for their server infrastructure. However, it is noted Docker is not intended to be used in the way we are applying it. As a result, it was decided that to better support container implementation of servers in the future and to support the transition from Virtualisation to Containerisation, a better, container-based solution for server management should be developed. The benefits of container-based servers are clear to see as a result of this research, but a more server-focused platform that can match the equivalent server-focused virtual machine based platforms that already exist is needed.