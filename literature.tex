%!TeX root=Dissertation.tex

\chapter{Some Literature}

Dario Taraborelli, has written a nice page illustrating some of \href{http://nitens.org/taraborelli/latex}{The Beauty of LaTeX} illustrating some of the finer points of TeX et.al. typesetting

\url{TeXample.net} is a site that has many spectacular examples\footnote{\url{http://www.texample.net/tikz/examples/}} of graphics creation in LaTeX. It mainly illustrates the use of the tikz package.

Two web sites provide forums for questions and answers on latex. \href{http://www.latex-community.org/}{LaTeX Community} and \href{http://tex.stackexchange.com/}{TeX - LaTeX Stack Exchange}

\section{References and searching}

\LaTeX\ and \BibTeX\ provide excellent support for referencing and citations (especially with natbib\footnote{see online manual at \url{http://mirror.ox.ac.uk/sites/ctan.org/macros/latex/contrib/natbib/natbib.pdf} or with the command \texttt{texdoc natbib}}).  Getting your .bib file populated with material can be time consuming. Among the many ways of doing this are:
\begin{itemize}
\item    Search engines like the University Library allow you to export your saved searched in a bibTeX file
\item    Google scholar https://scholar.google.co.uk/ have an option to create the BibTeX entry for results.
\item    Zotero provides a plugin for Firefox, that extracts information from a web page and exports a bibTeX file. (Great for getting references to Wikipedia, or getting book details from Amazon)
\end{itemize}

These are great, but sometimes the bibTeX file needs a little post-editing. Usually I end up deleting extraneous fields and escaping TeX characters if needed.

\begin{enumerate}
\item Search for articles, books etc, grab the bibTeX file,
\item \verb"\backslash cite{}"and use
\item \verb"\bibliography{save1,litrev,canon}" with
\item \verb"\bibliographystyle{plainnat}" (having \verb"\usepackage{natbib}" in the preamble)...
\end{enumerate}
 Harvard referencing of your material... job done

 \section{Writing your dissertation}
 
The most valuable document is the \emph{Project Handbook}, this gives guidance
on the structure and contents of the Terms-of-Reference and Dissertation.  It
also has the marking scheme, which is an important read as this describes what
we are looking for when marking the project.

\subsection{The Logbook}
Use your logbook and a project diary, make notes of what you do as you do
them.  As questions arise, make a note of them at the time, and then you are
ready for the weekly meeting.

In the meeting make notes of whet we discuss for the week ahead, and issues
that come up that can be looked at later.

Some students in the past have kept a project blog.

\subsection{The Dissertation} 
There are several good sites about with advice on how to write clearly.  Not
all of the advice is directly relevant to writing a B.Sc dissertation in
Computer Science.  The advice may not even be consistent, 

William Stallings \citep{Stallings} maintains a good website of resources for
Computer science students.

