%!TeX root=../Dissertation.tex
%!TeX bibfile=./synthesis.bib
%!TeX bibfile=./analysis.bib
\chapter{Conclusions}
%MAIN THING HERE IS COMPARING TO THE AIMS
\section{Main Test Result Conclusions}
The primary aim of this research was to measure performance between Containerised and Virtualised network infrastructure, with the hope of generating recommendations to either those that already use virtual machine infrastructure, or to those that are creating new headless servers and wondering which option to use.
With that in mind, I can confidently say that the total output from this work has been a success. A number of in-depth tests have been done to generate a solid set of results. These key and conclusive results are as follows:
\begin{itemize}
  \item Docker, when being used to run headless servers, performs faster than the same servers perform on a VMware machine.
  \item Docker can provide in most cases, over double the performance output when compared to VMware for the same workload.
  \item Docker uses far less system resources when compared directly with VMware.
  \item Docker's performance is more stable and reliable than VMware under load.
\end{itemize}

\section{Secondary Points of Interest}
%Talk about the ways that Docker is managed, and the ways that VMware is managed.
Other interesting points of discussion are also relevant, but less important than the main discoveries that are laid out above. One of these discussion points sits with the way that Docker and VMware are both managed. These two different methods are not exclusive to containerisation or virtualisation respectively, but when making recommendations based on the work done in this research, these points are still important to mention.
\begin{itemize}
  \item Docker is managed firstly from the command line. Dockerfiles can be written to quickly create new Docker Images, or, Images can be pulled directly from Dockerhub or a private repository. Images that are already created can be saved to .Tar files, or can be pushed to a repository. Multiple containers can use the same image, as IP addressing is done when launching a container. Networks which bridge to real interfaces must be created and defined by the user, using Docker's macvlan network driver. Macvlan is not the primary, or most supported network option provided by Docker.
  \item VMware is managed using VMware's workstation software. Virtual machines must either be created using the setup wizard using a real OS image file, cloned from another machine (If the original machine is deleted or lost, the clone will fail), or a direct copy of the VM can be created (by manually copying the Virtual Machine files). VMware comes ready with a `bridged' network, along with virtual networks that can connect with other services via adapters. The bridged network option is well supported by VMware.
\end{itemize}

I believe these are important points to mention as they should be considered by anyone that is planning on using this research to form an opinion about which technology to use. Neither of these methods is the right or wrong way of managing containers and virtual machines, but as they are different, it may still be an important basis for that persons decision making. For example, Docker's command line interface could for some end users such as network administrators be a feature they prefer over VMware's GUI interface. On the other hand, those with less technical knowledge might find VMware more user friendly. Another example is the way that Containers and Virtual Machines are stored. Both allow ample opportunity for backups, and rapid redeployment of server solutions should the need arise (though it could be argued that Docker is faster). These points are less quantitative, and more qualitative it is entirely up to the end user as to what matches their needs.

That being said, I do think that VMware's solution is more user friendly even if hardcore IT users may prefer a CLI over a more GUI. Someone with CLI knowledge may prefer that as an option, but they certainly wouldn't struggle to use VMware. The same could not be said the other way around. Docker is first and foremost a low resource solution, and as such should be used from the Command Line. Docker does have a `Docker Desktop' version that works on Windows and Mac, but this version uses different techniques to simulate the containers (for example, on windows it uses Windows Subsystem for Linux \citep{DockerWindows}), and as such we can't make recommendations for these versions of Docker, as there may be other constraints on container performance.

It is also important to talk here about the way that Networks are managed in both solutions. Docker's networks are generally configured by the user (a default virtual network is provided but not recommended). For containers to face a real network, the macvlan driver is used. This network mode isn't the most supported network option, and there can be problems regarding the handling of some IP traffic. This is evidenced by the issues around DHCP that we experienced in subsection \ref{Difficulties with Docker and DHCP}. When looking at VMware, the network management is far more robust. The systems that VMware uses to bridge virtual machines to real networks are far better supported, and for all intents and purposes, function exactly like a real machine would.


%Most of the issues come from having the hostmachine being one PC, results of covid and not having access to infrastructure at university. I was to do it again the most important step would be to run the test on a server machine on a real network. This test more a simulation than real life.

\section{Rounded Conclusion}

To sum up, Docker is a far better option than VMware when looking solely at performance metrics and resource usage. VMware was far slower, and as a result of it's high resource utilisation, it suffered fluctuation making it less reliable under load. However, Docker isn't designed around this kind of use, and as a result can suffer from being less straight and operable. VMware provides a more robust system for managing its instances. VMware's Bridged network mode is also more desirable than Docker's macvlan mode. 

It is clear that more work needs to be done to provide a suitable base for containerised networking specifically. The performance benefits to using Containers over Virtual Machines is a one that should be utilised, but Docker, as it stands now, probably isn't the best way forward. Should Docker want to provide this as a part of their platform in the future, they need to work on better ways to implement containers with real networks.

I am sure that once these issues are ironed out, that container based networking will be the future. I would hypothesise that we will start to see less and less virtualised infrastructure in favour of containers simply due to being able to get the same performance out of lesser hardware (something that will save costs).
%Actual conclusion is: Docker performs better than VMware, uses less resources, but there is still work to be done to make Docker more accessible, and to make it integrate with networks better. Once these issues are ironed out, containers will be the future.

\chapter{Recommendations}

\section{For those considering moving to Containers}
\label{sec:RecSMEs}
This section will comprise of information and recommendations to those that may want to use this research as part of their decision making process about moving to Containers to support their network infrastructure. Whether that be moving from virtual machines to container on an already existing machine, or whether it be part of a decision as to which technology a new system should be using.

The performance results in this report speak for themselves. Docker \emph{is} the superior choice if performance and reliability is the only metric that matters to you or your organisation. If there are no other considerations for your use-case, then the recommendation here would be that containerisation is absolutely the correct choice. However, it would be very unlikely that these are the only considerations. Security, Cost, Ease of Use, and Management Options are all important aspects to consider when deciding how to host network infrastructure.
\subsection{Security}
\label{conclandreccoSecurity}
Security in computing is often split into the CIA triad (Confidentiality, Integrity, Availability). Lets split both the technologies (Docker and VMware), along with the results we have collected, into the CIA triad, and see how they compare.
\subsubsection{Docker}
Confidentiality is something we didn't investigate during this research, but it has been suggested by other research that isolation is a problem with Docker. I would strongly recommend anyone planning on moving to Containers to read the paper by \citeauthor{watanda19} ``Emerging Trends, Techniques and Open Issues of
Containerization: A Review'' (\citeyear{watanda19}). This paper covers security and isolation concerns with containerisation which are important but outside of the scope of this research.

When we move to Integrity, we can comfortably say that this research tested this well, and we find that Docker's integrity stands up under scrutiny. The servers and services hosted via Docker were reliable and stable, and could hold up even when under heavy loads and stress testing.

Availability was also tested during this research. All the systems stayed up and running at reasonable levels throughout testing, with only minor hiccups that we would expect to see in any operational system anyway.

\subsubsection{VMware}
Again, we haven't touched on Confidentiality in our study as it was outside of our scope. However, VMware's isolation from the host machine, and between virtual machines is very well documented. For example; \citeauthor{VMwareIsolation} discusses this in their ``Survey of security isolation of virtualization, containers, and unikernels'' (\citeyear{VMwareIsolation}).

Where our research does interject is when looking at Integrity of data. The processing we saw in our tests, namely the database test (where important data would most likely be exchanged in a real system) showed that we had scattered results where we would observe very high latencies. As discussed in our evaluation, these were probably due to errors. The number of these errors we noticed was significantly more than that of Docker. So if you are planning on using VMware to host mission critical infrastructure, we would remind you that VMware may introduce these hitches that could be detrimental to the integrity of your data.

Availability is another area affected by VMware observed performance hits. The fluctuation caused by this lack of performance in our tests could result in end users sometimes not having access to systems during peak times, as sudden drops in performance happening quite often. This is not acceptable for any system that need to be used by a number of users. 

\subsubsection{Recommendation based on security}
Our recommendation, then, would be that should Confidentiality be the main priority of your system, Docker is probably not the solution for you. If Integrity and Availability of your data is more important to you though, then we would recommend Docker over VMware.

\subsection{Costing}
Another interesting point to mention here is cost of applications. VMware Pro (which is required to use the network manager) is part of a paid software package, whereas everything we did in our tests was possible using Docker's free plan. It is more than possible that however that businesses using Docker may want private repositories or other features that are only provided in Docker's paid plans. It is also possible, that an organisation using VMware, may not require access to the network manager, as bridged connections shouldn't require much (if any at all) configuration. The cost difference's between the two technologies at this point then simply comes down to the preference and requirements of the organisation setting out to use either of the technologies.

\subsection{Concluding recommendation}
From what is discussed in the sections above we can see that the use of VMware or Docker isn't clean-cut. It entirely comes down to the individual needs and requirements of the entity that is requiring a network solution. So whilst our study does a lot to prove the efficiency and performance benefits of containers, it cannot be used as the sole indicator of whether you should used Containers over Virtual Machines.

This moves us onto our next, and final section.

\section{Suggestions for further research and development}

%Recommendation up here first, then development opportunities (already written some of it) below that.
%Most of the issues come from having the hostmachine being one PC (you have talked about this above, this is just the recommendation, nothing else.

As discussed in the section above (Section \ref{sec:RecSMEs}), there are multiple reasons other than performance that come into the adoption of a new technology. Someone could develop the fastest database system in the world, but there would be very little uptake on the technology if it was also the least secure database system. End users need the reassurance that the systems they implement are right for them, and I don't yet feel fully confident in recommending container based networking, as I feel there are still issues that need to be addressed.

For full recommendation of Docker as a tool to host and run networks, it is clear there is work to be done. Firstly, Docker needs to do more work to offer flexibility in it's networking solutions. The macvlan adapter support is simply not good enough, or reliable enough to recommend as a long-term solution on a real network. Secondly, Docker should look to implement a more top-down management view of their containers, that allows end-users mass control over container networks in a similar way as is already possible on Virtualised networks. Furthermore, a deeper focus on the security implications would give end-users the confidence they need when hosting critical servers. Should these few sticking points be straightened I would be able to not only offer Docker as a better alternative in all ways to VMware, I would happily suggest that moving away from Virtualisation and into a new paradigm of Containerisation is inevitable. In-fact, despite the issues listed here, I am almost sure that containers are the future of single-machine networking.

Maybe the issue is that Docker was never designed to be the tool we are trying to turn it into. Docker was created firstly as a way for developers to easily host their applications for testing. It wasn't designed to be used as a permanent networking solution. With that in mind, my recommendation would be that a networking-focused container based solution need to be developed, from the ground up. I believe that a purpose-built containerised network solution, that addressed the issues discussed in this paper such as real world network cross-over, or security and isolation, would find themselves as a new leading industry standard. I also have no doubt that various Virtualisation and Containerisation solution providers are already working on these kinds of solutions. We are already starting to see uptake of containerisation in areas such as Platform as a Service (PaaS), whereby containers \emph{are} being used in an online environment, in order to provide cost-effective application environments \citep[Page 55]{PAASCloudBook}. %FINISH THIS TODO.
%FINISH THE BIT ABOVE
%FINISH THE BIT ABOVE
%FINISH THE BIT ABOVE
%FINISH THE BIT ABOVE