%!TeX root=TermsOfReference.tex

\section{Background}
Virtualisation has been utilised by a number of industries for a long time now, with first iterations of virtual machines dating back to IBM in the 1960s \citep{pugh95}. System Virtual machines allow an operating system to emulate the function of a full operating system layered on top of a base operating system. Functionally, this allows multiple different logical 'computers' with varying operating systems to run on one physical computer. In most modern implementations, virtualisation requires software (known as a hypervisor) to manage and create the virtual machines.

Whilst I was on a year-long placement provided as part of a sandwich course at my university, I had the chance to work in an IT risk department at a reputable enterprise in Newcastle Upon Tyne. whilst working there I witnessed first hand, infrastructure and operations departments using virtualisation for much of the internally and externally facing server infrastructure, in what was an unwieldy and cumbrous use of scripts to update and install patches and dependencies across a multitude of systems. This resulted in a number of incidents where systems had to be pulled down in order to update the Operating Systems of individual virtual machines. These methods often caused unnecessary down time for systems, resulting in substantial risk for the business. Furthermore, this method often put a substantial stress on the hardware of these systems, with hardware boxes often running at full resource potential under heavy load, and whilst these boxes were designed to withstand these kinds of loads, it still affected the longevity of the hardware.

In recent years there has been research into the use of containers instead of virtual machines for various operations across computing industries \citep{watanda19}. Containers are different from virtual machines in that they run on the base OS, and don't require a secondary emulated operating system to be managed by a hypervisor. This is a benefit as it reduces the resources \citep{joy15} used by each instance. Overall, containers are a more lightweight and efficient system, that are easier to keep up to date. Whilst research has claim in displaying the benefits of containerisation, much of the server infrastructure of enterprise remains reliant on Virtualisation, not Containerisation.

I have also had the opportunity to partake in gatherings held by CoTech, a network of tech co-operatives that meet semi-regularly to discuss various tech related topics. Whilst I was there, it was discussed that a large portion of these small enterprises used Docker, a system for packaging and deploying containers, to develop applications for their clients.

There is historical research that compares virtualisation and containerisation for other technical applications, for example \citep{dua14} compares the two methods inside the context of PaaS (Platform as a service), which is similar to co-techs use of Docker. It seems that the application for containerisation has been realised as early as 2014 when it is being applied to the development and hosting of applications, but not as much when talking about the running of wider server infrastructure. This is further supported by research from '451 Research' who in 2017 estimated a compound annual growth rate of 40\% for containers in 2020 \citep{451}, suggesting a coming paradigm shift from virtualisation to containerisation.

\section{Proposed Work}
\label{proposed}
I plan to do similar work to the studies I have already mentioned \citep{joy15}, \citep{dua14}, but instead focus this work on my own context and my own interest in Operating Systems and Server Infrastructure. I have experience with running headless servers on virtual machines already, through the Advanced Operating Systems module I did in my second year of study at Northumbria University. 

It is important to set a baseline system here, and to best reflect a realistic topology, I will use LAMP (Linux + Apache + MySQL + PHP). I have experience with Ubuntu server, so I will use this version of Linux as the base operating system for my virtual machines, and plan to host a full working topology of servers, including a DNS architecture, a web-server architecture that includes HTTP (Apache) servers, NFS servers and MySQL servers. The reasoning for doing this is to create as realistic a topology as possible, and to ensure a somewhat realistic level of network traffic so that meaningful data can be captured.

For the virtual machines, I will use the virtual machine infrastructure available to me through the university, and for the container infrastructure, I plan on using Docker. This is because Docker is a popular choice among app developers, and is supposed to make container deployment easy. Though, as I have never worked with containers before, if Docker fails to be a good way of providing containers for web-servers, then there are a number of other ways to deploy containers, such as LXC, that I could turn to as a contingency.

I will also need to set a standard for measurement to ensure that my data is meaningful and uses the scientific method. I plant to use tools such as iPerf3 (for network performance) and Sysbench (for hardware performance) across a number of servers. It is important to measure both network and hardware performance (CPU, Memory, Disk, etc) to ensure that described  benefits are achieved for the system as a whole. iPerf is an industry standard tool for measuring network performance on Linux systems, whilst Sysbench is a full benchmarking suite for linux that will let me benchmark the CPU, file IO, and importantly, MySQL performance (allowing direct measurement of database performance on the machine that will be hosting the previously mentioned MySQL database. Whilst I have mentioned these benchmarking tools, it is possible that they could have problems with integrating with certain applications or environments, in this case, there is a number of other standard benchmark utilities (Phoronix Test Suite, KDiskMark, UnixBench, etc) available that should give me the flexibility to create measurements. Whilst these utilities all have differing overheads within them (meaning they will use up varying resources to run the benchmark), as long as I use the same benchmark across the same machines I am comparing, this overhead shouldn't effect the measurable performance difference between these machines.

Once data is collected, I will do a comparative analysis of the data to determine if there is a clear improvement as previous research suggests, and if this difference is enough to justify a change in the current best-practice use of virtualisation.

\section{Aims and Objectives}

\subsection{Aims}
\begin{quote}

To compare and contrast the performance difference and hardware impact between virtualisation and containerisation when running headless servers.
%analysis section Chapters: what is network infrastructure I want to model (analysis chapter), design work of the two implementations. Could include creation/justification of benchmark (made one or found one)\

%Synthesis chapters: Build the systems, run and gather the data, analyse the data.

%Lift sections from handbook for the evaluation.
\end{quote}

\subsection{Objectives}
Your objective list is a series of measurable objectives, can you tick each one off as \emph{done}?  I usually expect between 8 and 12 objectives

\begin{enumerate}
	\item \textbf{Explain the problem domain that encompasses current practices in virtualisation.}
	\item \textbf{Explanation of the two different methods in practice.}
	\item \textbf{Determine the software and hardware to be used.}
	\item \textbf{Design the network infrastructure to be built.}
	\item \textbf{Build the network topology on both the virtual machines and containers.}
	\item \textbf{Determine a method to scientifically evaluate and determine performance of both systems.}
	\item \textbf{Accurately measure performance of the two systems.}
	\item \textbf{Compare and contrast the findings between the findings for both systems to determine improvements or failings.}
	\item \textbf{Create a recommendation regarding the real-world implementation of containerisation in this use case.}

\end{enumerate}

\section{Skills}
This is where you can cover the skills you have relevant to the project and the new skills you are going to acquire during the project.
\begin{enumerate}
	\item Advanced Operating Systems 1, see module KF5004.
	\item Computer Networking Experience.
	\item Computer Technology, see module KF4004.
	\item Virtual machine configuration.
	\item LAMP topology experience.
	\item Docker configuration.
\end{enumerate}

\section{Resources}
I will require access to Virtual Machine infrastructure in order to do build my topology and compare virtual machine performance.
I will also require access to a machine that has the same hardware and system resources as those shared by the virtual machines, as I will need to perform my containerisation tests on the same hardware in order to effectively control that variable.

\subsection{Hardware}
I shouldn't require any hardware as long as I can get remote access to the virtual machine infrastructure in a way that allows me to use the same resources for containerisation (as mentioned above).

As contingency, if I cant access this infrastructure, I will still need some way of creating and managing virtualisation. This might be possible on my own hardware, though I'd rather use university infrastructure as this should be more reliable and accessible.

\subsection{Software}
I will need to install Docker, and will need access to benchmarking suites, but these are free and/or open-source.

\section{Structure and Contents of the Report}
Below I will set out the chapters that I will most likely have in my final report.
\subsection{Report Structure}

\paragraph{Introduction}  Sets out the background and motivation for the project.  Summarises the work done, the results, the conclusions, and the recommendations for future work.  It is a one chapter summary of the \emph{entire} project.

\paragraph{Defining the problem}  Objective \ref{understand-problem} requires a precise definition of the problem you are solving.  Don't forget to reference good source material  See section \ref{proposed}.

\paragraph{Possible Solutions} Discuss the possible solutions, compare the
alternatives, and select the one to use for the  implementation.

\subsection{List of Appendices}
What Appendices you will include.  A copy of the TOR should be the first, followed by the Ethics form and the Risk Assessment.

Others might include design documentation, code listings, tables of results (if too large to include in the main text).

\section{Marking Scheme}
The marking scheme sets out what criteria we are going to use for the project.

\paragraph{Project Type} General Computing or Software Engineering projects

\paragraph{Project Report}  State which chapters constitute the \emph{Analysis}, the \emph{Synthesis}, and the \emph{Evaluation}.  This help me when marking to know when to stop reading one section and put a mark down for it.

\paragraph{Product}  List the deliverables that make up the \emph{Product}.  Code, design, requirements specifications, test plans, etc.

For the \emph{Fitness for Purpose} and \emph{Build Quality}  list the critera used to asses the product by

\subparagraph{Fitness for Purpose}~
\begin{itemize}
	\item meet requirements identified
	\item other appropriate measures
\end{itemize}

\subparagraph{Build Quality}~
\begin{itemize}
	\item Requirements specification and analysis
	\item Design Specification
	\item Code quality
	\item Test plan and Results
\end{itemize}

\clearpage

\section{Project Plan}
\noindent
\rotatebox{90}{%!TeX root=TermsOfReference.tex

% A lot of the settings here are tuned to fit a landscape gantt chart into
% an A4 piece of paper.
\begin{ganttchart}[
time slot format=little-endian,
calendar week text=\currentweek,
x unit=2.4pt,
y unit chart=14pt,
y unit title=12pt,
title label font=\scriptsize,
bar top shift=.15,
bar height=0.7,
milestone label font = \small,
group label font = {\tiny\bfseries},
group inline label node/.append style=centered,
hgrid=true,vgrid={*6{draw=none},dotted},
region/.style={inline,group peaks width=2,
  group peaks height=0.25, group height=0.5,
  group top shift=0.2 ,group/.append style={fill=#1}},
milestone left shift=0,
milestone right shift=1,
ms/.style={inline,
    milestone inline label node/.append style={#1=0pt}}
]%
% For semester dates see
% https://www.northumbria.ac.uk/about-us/university-services/academic-registry/registry-records-and-returns/academic-calendars/
{21/9/20}{28/05/21} %<- Dates Gantt Chart runs from and to

\gantttitlecalendar{year,month,week=1}\\

% Highlight Smesters and Vactions
\ganttgroup[region=blue!20]{Semester 1}{21/9/20}{22/1/21}
\ganttgroup[region=blue!50]{Semester 2}{18/1/21}{28/5/21}\\
\ganttgroup[region=red!50]{Christmas}{21/12/20}{8/1/21}
\ganttgroup[region=green!25]{Easter}{29/3/21}{16/4/21}\\

% Project Deadlines (from the Project Handbook)
\ganttmilestone[ms=left]{PID}{12/10/20}
\ganttmilestone[ms=left]{final \bfseries TOR}{16/11/20}
\ganttmilestone[ms=right]{Draft Analysis}{30/11/20}
\ganttmilestone[ms=left,milestone/.append style={fill=red}]{\bfseries Submit}{29/4/21}
\ganttnewline[thick]

% --Tasks go here
% put in a title, a start date, end date...
\ganttbar{TOR}{28/9/20}{26/10/20}
\ganttbar[inline]{\emph{revise}}{28/10/20}{16/11/20}\\
\ganttbar{Analysis}{30/9/20}{24/11/20}\\
\ganttbar{Design}{31/10/20}{19/1/21}
% Note how you can refer to labels set in the objectives list!
\ganttmilestone[ms=right]{Build complete (Obj \ref{write-code})}{18/1/20}\\
\ganttbar{Exploration}{22/10/19}{15/1/20}
\end{ganttchart}

}
